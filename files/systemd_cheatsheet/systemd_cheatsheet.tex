\documentclass[a4paper]{article}

\usepackage[table]{xcolor}
\usepackage{tabularx}
\usepackage[
	     citecolor=red,backref,bookmarks,
             pdfpagemode=None,
             pdftitle={Systemd Cheatsheet},
             pdfauthor={PetrR},
             pdfsubject={Systemd Cheatsheet},
             pdfkeywords={Systemd Cheatsheet, Refcard, Cheat Sheet, Systemd},
             pdfview=FitB
]{hyperref}

\title{Systemd Cheatsheet}

\pagestyle{empty}

\addtolength{\voffset}{-3.5cm}
\addtolength{\textheight}{7cm}
\addtolength{\hoffset}{-3.7cm}
\addtolength{\textwidth}{7.4cm}

% -----------------------------------------------------------------------

\begin{document}
\begin{center}

{\huge Systemd Cheatsheet}

\vspace{8mm}
\small

\begin{tabularx}{\textwidth}{ |l|X|X| }
\hline
\rowcolor[gray]{.8}
\bfseries Sysvinit Command    & \bfseries Systemd Command                                                             & \bfseries Notes \\\hline
\tt service httpd start       & \tt systemctl start httpd.service                                                     & Used to start a service (not reboot persistent). \\\hline
\tt service httpd stop        & \tt systemctl stop httpd.service                                                      & Used to stop a service (not reboot persistent). \\\hline
\tt service httpd restart     & \tt systemctl restart httpd.service                                                   & Used to stop and then start a service. \\\hline
\tt service httpd reload      & \tt systemctl reload httpd.service                                                    & When supported, reloads the config file without interrupting pending operations. \\\hline
\tt service httpd condrestart & \tt systemctl condrestart httpd.service                                               & Restarts if the service is already running. \\\hline
\tt service httpd status      & \tt systemctl status httpd.service                                                    & Tells whether a service is currently running. \\\hline
\tt service --status-all      & \tt systemctl list-units --type service                                               & Displays the status of all services. \\\hline
\tt ls /etc/rc.d/init.d/      & \tt systemctl list-unit-files --type=service                                          & Used to list the services that can be started or stopped. Used to list all the services and other units. \\\hline
\tt chkconfig httpd on        & \tt systemctl enable httpd.service                                                    & Turn the service on, for start at next boot, or other trigger. \\\hline
\tt chkconfig httpd off       & \tt systemctl disable httpd.service                                                   & Turn the service off for the next reboot, or any other trigger. \\\hline
\tt chkconfig httpd           & \tt systemctl is-enabled httpd.service                                                & Used to check whether a service is configured to start or not in the current environment. \\\hline
\tt chkconfig --list          & \tt systemctl list-unit-files --type=service \newline ls /etc/systemd/system/*.wants/ & Print a table of services that lists which runlevels each is configured on or off. \\\hline
\tt chkconfig httpd --list    & \tt ls /etc/systemd/system/*.wants/httpd.service                                      & Used to list what levels this service is configured on or off. \\\hline
\tt chkconfig httpd --add     & \tt systemctl daemon-reload                                                           & Used when you create a new service file or modify any configuration. \\\hline
\end{tabularx}

\vspace{6mm}

\begin{tabularx}{\textwidth}{ |l|l|X| }
\hline
\rowcolor[gray]{.8}
\bfseries Sysvinit Runlevel & \bfseries Systemd Target                              & \bfseries Notes \\\hline
0                           & runlevel0.target, poweroff.target                     & Halt the system. \\\hline
1, s, single                & runlevel1.target, rescue.target                       & Single user mode. \\\hline
2, 4                        & runlevel2.target, runlevel4.target, multi-user.target & User-defined/Site-specific runlevels - identical to 3. \\\hline
3                           & runlevel3.target, multi-user.target                   & Multi-user, non-graphical. Users can usually login via multiple consoles or via the network. \\\hline
5                           & runlevel5.target, graphical.target                    & Multi-user, graphical. Usually has all the services of runlevel 3 plus a graphical login. \\\hline
6                           & runlevel6.target, reboot.target                       & Reboot \\\hline
emergency                   & emergency.target                                      & Emergency shell \\\hline
\end{tabularx}

\vspace{6mm}

\begin{tabularx}{\textwidth}{ |l|X| }
\hline
\rowcolor[gray]{.8}
\bfseries Command                                       & \bfseries Notes \\\hline
\tt systemctl get-default                               & Determine which target unit is used by default. \\\hline
\tt systemctl set-default multi-user.target             & Change default boot target to multi-user.target. \\\hline
\tt journalctl -b                                       & Show all messages from this boot. \\\hline
\tt journalctl -b -p err                                & Show all messages of priority levels ERROR (4) and worse, from the current boot. \\\hline
\tt journalctl -p warning --since="2014-06-14 23:59:59" &  View the warning or higher priority messages from certain point in time. \\\hline
\tt journalctl -f                                       & Follow new messages. \\\hline
\tt journalctl /usr/sbin/httpd                          & Show all messages by a specific executable. \\\hline
\tt journalctl --full                                   & Display full (= not truncated) messages. \\\hline
\tt systemctl --state=failed                            & Lets find the systemd services which fail to start. \\\hline
\tt systemctl list-units --type=target                  & Show current runlevel. \\\hline
\tt systemctl isolate graphical.target                  & Changes the current target (runlevel). \\\hline
\tt systemctl rescue/emergency                          & Changing to Rescue(single user mode)/Emergency Mode. \\\hline
\tt systemd-cgls                                        & cgroup tree \\\hline
\tt systemctl show -p "Wants" multi-user.target         & What other units does a unit depend on? \\\hline
\tt systemctl list-jobs                                 & Check for possibly stuck jobs use. \\\hline
\end{tabularx}

\vspace{6mm}

\begin{tabularx}{\textwidth}{ |X|X|X| }
\hline
\rowcolor[gray]{.8}
\bfseries Old Command & \bfseries Systemd Command  & \bfseries Description \\\hline
\tt halt              & \tt systemctl halt         & Halts the system. \\\hline
\tt poweroff          & \tt systemctl poweroff     & Powers off the system. \\\hline
\tt reboot            & \tt systemctl reboot       & Restarts the system. \\\hline
\tt pm-suspend        & \tt systemctl suspend      & Suspends the system. \\\hline
\tt pm-hibernate      & \tt systemctl hibernate    & Hibernates the system. \\\hline
\tt pm-suspend-hybrid & \tt systemctl hybrid-sleep & Hibernates and suspends the system. \\\hline
\end{tabularx}


% Footer
\vfill \hrule\smallskip
{\tiny 
This card may be freely distributed under the terms of the GNU general public licence
}

\end{center}
\end{document}
